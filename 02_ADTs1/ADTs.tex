\documentclass{beamer}
\usetheme[progressbar=frametitle]{metropolis}
\setbeamertemplate{frame numbering}[fraction]
\useoutertheme{metropolis}
\useinnertheme{metropolis}
\usefonttheme{metropolis}

% Barra de progreso más ancha
\makeatletter
\setlength{\metropolis@titleseparator@linewidth}{2pt}
\setlength{\metropolis@progressonsectionpage@linewidth}{2pt}
\setlength{\metropolis@progressinheadfoot@linewidth}{2pt}
\makeatother

\usepackage[utf8]{inputenc}
\usepackage[english]{babel}
\usepackage[compatibility=false]{caption}
\usepackage[listings,theorems]{tcolorbox}
\usepackage{minted}

\usemintedstyle{friendly}

\DeclareCaptionFont{captionfont}{\fontsize{8}{8}\selectfont}
\captionsetup{font=captionfont}

\title{Programación con tipos dependientes en Idris 2}
\subtitle{ADTs I: Bool}
\author{Axel Suárez Po}
\institute{BUAP}
\date{\today}

\begin{document}

\begin{frame}
  \titlepage
\end{frame}

\begin{frame}[t]
  \frametitle{Contenidos}
  \tableofcontents
\end{frame}

\section{Introducción}

\begin{frame}[fragile]
  \frametitle{Una pregunta interesante}

  \Large{¿Cuántas funciones \emph{distintas} hay de \mintinline{idris}|Bool -> Bool|?}
  \vspace{10pt}

  \begin{listing}[H]
    \begin{center}
      \begin{minipage}{0.35\textwidth}
          \begin{minted}[fontsize=\scriptsize, autogobble, stripnl=false]{idris}
          f : Bool -> Bool
          f b = ?r

          \end{minted}
      \end{minipage}
      \hspace{-0.05\textwidth}
      \vrule
      \hspace{0.025\textwidth}
      \begin{minipage}{0.35\textwidth}
          \begin{minted}[fontsize=\scriptsize, autogobble, breaklines]{java}
          boolean f(boolean b) {
            return ?;
          }
          \end{minted}
      \end{minipage}
    \end{center}
    \caption{Bool a Bool en Idris y Java}
    \label{lst:btobIdrisJava}
  \end{listing}


  \footnotetext[1]{\scriptsize{Trate de contestar en menos de 10 segundos}}
\end{frame}

\begin{frame}[fragile]
  \frametitle{Las funciones obvias}

  \begin{listing}[H]
    \begin{center}
      \begin{minipage}{0.42\textwidth}
          \begin{minted}[fontsize=\scriptsize, autogobble, stripnl=false]{idris}
          -- La función not
          notb : Bool -> Bool
          notb True  = False
          notb False = True



          -- La función identidad
          idb : Bool -> Bool
          idb True  = True
          idb False = False


          \end{minted}
      \end{minipage}
      \hspace{-0.05\textwidth}
      \vrule
      \hspace{0.025\textwidth}
      \begin{minipage}{0.45\textwidth}
          \begin{minted}[fontsize=\scriptsize, autogobble, stripnl=false]{java}
          boolean notb(boolean b) {
            if (b == true)
              return false;
            else
              return true;
          }

          boolean idb(boolean b) {
            if (b == true)
              return true;
            else
              return false;
          }
          \end{minted}
      \end{minipage}
    \end{center}
    \caption{Funciones Bool a Bool obvias.}
    \label{lst:btobIdrisJavaObvias}
  \end{listing}


  \footnotetext[1]{\scriptsize{Claramente se pueden escribir de forma más concisa.}}
\end{frame}

\begin{frame}[fragile]
  \frametitle{Las funciones no tan obvias}

  \begin{listing}[H]
    \begin{center}
      \begin{minipage}{0.5\textwidth}
          \begin{minted}[fontsize=\scriptsize, autogobble, stripnl=false]{idris}
            -- La función constante True
            constTrue : Bool -> Bool
            constTrue True  = True
            constTrue False = True



            -- La función constante False
            constFalse : Bool -> Bool
            constFalse True  = False
            constFalse False = False


          \end{minted}
      \end{minipage}
      \hspace{-0.05\textwidth}
      \vrule
      \hspace{0.025\textwidth}
      \begin{minipage}{0.45\textwidth}
          \begin{minted}[fontsize=\scriptsize, autogobble, stripnl=false]{java}
            boolean constTrue(boolean b) {
              if (b == true)
                return true;
              else
                return true;
            }

            boolean constFalse(boolean b) {
              if (b == true)
                return false;
              else
                return false;
            }
          \end{minted}
      \end{minipage}
    \end{center}
    \caption{Funciones Bool a Bool no tan obvias.}
    \label{lst:btobIdrisJavaNoObvias}
  \end{listing}

  \footnotetext[1]{\scriptsize{¿Serán todas? ¿Cómo estar seguro?}}
\end{frame}

\begin{frame}[fragile]
  \frametitle{Una pregunta más interesante}

  \large{¿Cuántas funciones hay de \mintinline{idris}|Bool -> Bool -> Bool|?}
  \vspace{10pt}

  \begin{listing}[H]
    \begin{center}
      \begin{minipage}{0.45\textwidth}
          \begin{minted}[fontsize=\scriptsize, autogobble, stripnl=false]{idris}
          f : Bool -> Bool -> Bool
          f a b = ?r

          \end{minted}
      \end{minipage}
      \hspace{-0.05\textwidth}
      \vrule
      \hspace{0.025\textwidth}
      \begin{minipage}{0.53\textwidth}
          \begin{minted}[fontsize=\scriptsize, autogobble, breaklines]{java}
          boolean f(boolean a, boolean b) {
            return ?;
          }
          \end{minted}
      \end{minipage}
    \end{center}
    \caption{Dos Bool's a Bool en Idris y Java}
    \label{lst:2boolIdrisJava}
  \end{listing}
\end{frame}

\section{Tipos de datos algebraicos}
\subsection{Alternativas}
\subsection{Pattern matching en Alternativas}

\begin{frame}[fragile]
  \frametitle{¿Qué es un tipo de datos algebraico?}

  \begin{block}{ADT}
    \vspace{8pt}
    Un tipo de datos algebraico (ADT) es un tipo de datos \textbf{compuesto},
    esto quiere decir que se forma combinando otros tipos de datos.

    Las dos clases más comunes de tipos de datos compuestos utilizados son:
    \begin{itemize}
      \item \textbf{Productos:} Los valores de este tipo son combinaciones de valores de
        los tipos que lo componen. (\textbf{Tuplas})
      \item \textbf{Sumas:} Los valores de este tipo alternan entre tipos distintos de
        forma excluyente. (\textbf{Either})
    \end{itemize}
  \end{block}

\end{frame}

\begin{frame}[fragile]
  \frametitle{Sumas}

  Los \textbf{tipos suma} (\emph{sum types}) también suelen nombrarse como
  \textbf{coproductos, uniones disjuntas (etiquetadas), variantes o alternativas.}

  La sintaxis para definir un tipo suma, es separando sus \textbf{constructores} con
  barras verticales (|).

  Un ejemplo de estos, es un tipo bastante conocido: \mintinline{idris}|Bool|.

  \begin{listing}[H]
    \begin{center}
      \begin{minipage}{0.5\textwidth}
        \begin{minted}[autogobble]{idris}
          data Bool = False | True
        \end{minted}
      \end{minipage}
    \end{center}
    \caption{Definición de Bool de la librería estándar.}
    \label{lst:booldefinition}
  \end{listing}

  \footnotetext[1]{\scriptsize{Más adelante se definirá con precisión a que se refiere
  la palabra \emph{constructores}.}}
\end{frame}

\begin{frame}[fragile]
  \frametitle{Sumas: Construcción}

  Para \emph{construir} un valor que pertenezca a este tipo se hace uso de sus
  constructores:

  \begin{listing}[H]
    \begin{center}
      \begin{minipage}{0.5\textwidth}
        \begin{minted}[autogobble,fontsize=\scriptsize]{idris}
          data Bool = False | True

          x : Bool
          x = True

          x' : Bool
          x' = False
        \end{minted}
      \end{minipage}
    \end{center}
    \caption{Valores del tipo Bool.}
    \label{lst:boolvalues}
  \end{listing}

  \footnotetext[1]{\scriptsize{Nótese otra vez que se está siendo \emph{vago}
  con la definición de qué quiere decir \textbf{construir}.}}
\end{frame}

\begin{frame}[fragile]
  \frametitle{Sumas: Utilización}

  Cuando se tiene una expresión de un tipo suma, hay una cantidad de cosas limitadas
  que se pueden hacer:

  \begin{itemize}
    \item Ignorar el valor.
    \item Colocarlo en \textbf{uno o más lugares} donde se espera una
      expresión de ese tipo.
      \begin{itemize}
        \item Pasarlo como parámetro
        \item Referenciarlo en expresiones con el mismo tipo
      \end{itemize}
    \item \emph{Deconstruirlo} con \textbf{múltiples} \emph{patrones de constructor},
      utilizando el \textbf{pattern matching}.
  \end{itemize}

  Las dos primeras opciones son comunes a todas las expresiones, pero la
  última es única de los tipos suma.
\end{frame}

\begin{frame}[fragile]
  \frametitle{Pattern matching}

  Cuando se utiliza \textbf{pattern matching}, se busca ``encajar'' (\emph{match}) un
  valor con ciertos patrones, pudiendo asignar variables cuando este \emph{match}
  es exitoso.

  Este pattern matching puede ocurrir en diferentes posiciones, pero nos enfocaremos por
  el momento en la \textbf{definición de funciones}.

  El pattern matching puede tomar varias formas:
  \begin{itemize}
    \item Patrón de literales
    \item \textbf{Patrón de constructor}
    \item Patrón de variable y patrón de descartamiento (\emph{catch all})
    \item Patrones especializados: \textbf{as-patterns, dot-patterns, view-patterns,
      with-patterns, etc}.
  \end{itemize}

\end{frame}

\begin{frame}[fragile]
  \frametitle{Pattern matching: Patrón de constructor}

  El pattern matching en el que centraremos nuestra atención es el
  \textbf{patrón de constructor.}

  Tal como su nombre lo indica, consiste en enumerar los constructores del tipo
  que se está operando:

  \begin{listing}[H]
    \begin{center}
      \begin{minipage}{0.42\textwidth}
          \begin{minted}[fontsize=\scriptsize, autogobble, stripnl=false]{idris}
          data Bool = False | True

          -- La función not
          notb : Bool -> Bool
          notb True  = False
          notb False = True

          -- La función identidad
          idb : Bool -> Bool
          idb True  = True
          idb False = False
          \end{minted}
      \end{minipage}
    \end{center}
    \caption{Patrón de constructor}
    \label{lst:matchconstructor}
  \end{listing}

\end{frame}

\begin{frame}[fragile]
  \frametitle{Pattern matching: Patrón de variable}

  Sin embargo, la función identidad se puede definir de una manera más sencilla.

  Para simplemente capturar el valor se puede utilizar un identificador (que no se
  encuentre en otro patrón):


  \begin{listing}[H]
    \begin{center}
      \begin{minipage}{0.42\textwidth}
          \begin{minted}[fontsize=\scriptsize, autogobble, stripnl=false]{idris}
          data Bool = False | True

          idb : Bool -> Bool
          idb True  = True
          idb False = False

          idb' : Bool -> Bool
          idb' b = b

          -- Cualquier identificador válido
          idb'' : Bool -> Bool
          idb'' nombreLargo = nombreLargo
          \end{minted}
      \end{minipage}
    \end{center}
    \caption{Patrón de variable}
    \label{lst:matchvariable}
  \end{listing}

\end{frame}

\begin{frame}[fragile]
  \frametitle{Pattern matching: Patrón de descartamiento}

  Existen ocasiones en las que no es necesario utilizar un parámetro, como
  en las funciones constantes que se encontraron con anterioridad:


  \begin{listing}[H]
    \begin{center}
      \begin{minipage}{0.42\textwidth}
          \begin{minted}[fontsize=\scriptsize, autogobble, stripnl=false]{idris}
          data Bool = False | True

          constTrue : Bool -> Bool
          constTrue True  = True
          constTrue False = True

          -- Otra alternativa
          constTrue'' : Bool -> Bool
          constTrue'' b = True

          -- Una forma más concisa
          constTrue' : Bool -> Bool
          constTrue' _ = True
          \end{minted}
      \end{minipage}
    \end{center}
    \caption{Patrón de descartamiento}
    \label{lst:matchvariable}
  \end{listing}

\end{frame}

\begin{frame}[fragile]
  \frametitle{Pattern matching: Catch-all}

  Los dos tipos anteriores de pattern matching (variable y descartamiento) tienen
  un comportamiento que se conoce como \textbf{catch-all}.

  Esto quiere decir que cuando se encuentra uno  de estos, los siguientes patrones
  sobre este mismo parámetro se vuelven inalcanzables.

  \begin{listing}[H]
    \begin{center}
      \begin{minipage}{0.42\textwidth}
          \begin{minted}[fontsize=\scriptsize, autogobble, stripnl=false]{idris}
          -- Uso incorrecto de un patrón catch-all
          -- compila con advertencias
          constTrue'alt : Bool -> Bool
          constTrue'alt _     = True
          constTrue'alt False = False
          \end{minted}
      \end{minipage}
    \end{center}
    \caption{Patrón no alcanzable}
    \label{lst:matchunreachable}
  \end{listing}
\end{frame}

\begin{frame}[fragile]
  \frametitle{Totalidad}

  Los patrones catch-all son particularmente útiles para asegurar la
  \textbf{totalidad} de las funciones.

  Por ejemplo, el tipo de datos \mintinline{idris}|Char| lo podemos conceptualizar como
  un tipo suma de todos los carácteres posibles, por lo que una función
  \mintinline{idris}|Char -> String| requeriría bastantes patrones.

  \begin{listing}[H]
    \begin{center}
      \begin{minipage}{0.42\textwidth}
          \begin{minted}[fontsize=\scriptsize, autogobble, stripnl=false]{idris}
          -- Esta definición solo es conceptual
          data Char = 'a' | 'b' | ...

          inicialANombre : Char -> String
          inicialANombre 'a' = "Axel"
          inicialANombre 'i' = "Iván"
          inicialANombre 'j' = "Jesús"
          inicialANombre 's' = "Samuel"
          inicialANombre  _  = "Otro"
          \end{minted}
      \end{minipage}
    \end{center}
    \caption{Catch-all para función total}
    \label{lst:catchalltotal}
  \end{listing}
\end{frame}

\begin{frame}[fragile]
  \frametitle{Parcialidad}

  Si no se han cubierto todas las alternativas y además no se incluye un patrón catch
  all, entonces se dice que la función es \textbf{parcial}.

  Existen dos casos en los que se tienen funciones parciales:
  \begin{itemize}
    \item \textbf{Error del programador:} Al programador le faltó considerar alternativas
      para la función, por lo que deberá agregar un patrón catch-all o manejar los demás
      casos.
    \item \textbf{Se requiere una función parcial:} En el casos de que realmente se
      requiera una función parcial, Idris requiere que se anote explícitamente como tal
      con la palabra clave \mintinline{idris}|partial|.
  \end{itemize}

\end{frame}

\begin{frame}[fragile]
  \frametitle{GADT's}

  La notación de GADT's es una alternativa para definir tipos de datos de una forma más
  explícita.

  Además, permite definir tipos de datos \textbf{indexados} y \textbf{parametrizados}.

  El tipo de datos \mintinline{idris}|Bool| también se puede definir como sigue:

  \begin{listing}[H]
    \begin{center}
      \begin{minipage}{0.42\textwidth}
          \begin{minted}[fontsize=\scriptsize, autogobble, stripnl=false]{idris}
          data Bool : Type where
            False : Bool
            True  : Bool

          -- La definición anterior
          data Bool = False | True
          \end{minted}
      \end{minipage}
    \end{center}
    \caption{GADT para Bool}
    \label{lst:catchalltotal}
  \end{listing}
\end{frame}

\begin{frame}[fragile]
  \frametitle{GADT's}

  Aparte de ser más explícita y tener mayores capacidades, también permite ver
  claramente que es lo que implica definir un tipo de datos:
  \begin{itemize}
    \item \mintinline{idris}|Bool| es un tipo
    \item \mintinline{idris}|False| es del tipo \mintinline{idris}|Bool|
    \item \mintinline{idris}|True| es del tipo \mintinline{idris}|Bool|
    \item \textbf{No hay otra cosa que sea un} \mintinline{idris}|Bool|
  \end{itemize}
  Esta última regla es lo que garantiza que un pattern matching exhaustivo implique
  la totalidad de una función.

  \begin{listing}[H]
    \begin{center}
      \begin{minipage}{0.42\textwidth}
          \begin{minted}[fontsize=\scriptsize, autogobble, stripnl=false]{idris}
          data Bool : Type where
            False : Bool
            True  : Bool
          \end{minted}
      \end{minipage}
    \end{center}
    \caption{GADT para Bool}
    \label{lst:catchalltotal}
  \end{listing}
\end{frame}

\begin{frame}[fragile]
  \frametitle{Tipos suma, otra vez.}

  Se ha mencionado que a estos se les conoce como tipos suma, pero no se ha mostrado
  un ejemplo de suma de tipos.

  A continuación se muestra una:

  \begin{listing}[H]
    \begin{center}
      \begin{minipage}{0.42\textwidth}
          \begin{minted}[fontsize=\scriptsize, autogobble, stripnl=false]{idris}
          data RGB = Red | Green | Blue
          data CMY = Cyan | Magenta | Yellow

          data Color = R RGB | C CMY
          \end{minted}
      \end{minipage}
    \end{center}
    \caption{ADT para Colores}
    \label{lst:catchalltotal}
  \end{listing}

  Como se puede observar, se han definido dos tipos de datos y se ha creado una suma de
  ellos.
\end{frame}

\begin{frame}[fragile]
  \frametitle{Tipo suma: Colores}

  Si se quisiera definir una función sobre este tipo de datos, se puede realizar
  pattern matching recursivo sobre sus constructores:

  \begin{listing}[H]
    \begin{center}
      \begin{minipage}{0.42\textwidth}
          \begin{minted}[fontsize=\scriptsize, autogobble, stripnl=false]{idris}
          complemento : Color -> Color
          complemento (R Red) = C Cyan
          complemento (R Green) = C Magenta
          complemento (R Blue) = C Yellow
          complemento (C Cyan) = R Red
          complemento (C Magenta) = R Green
          complemento (C Yellow) = R Blue
          \end{minted}
      \end{minipage}
    \end{center}
    \caption{Función sobre tipo de datos más complejo}
    \label{lst:funcioncolor}
  \end{listing}
\end{frame}

\begin{frame}[fragile]
  \frametitle{Tipo suma: Bool}

  \mintinline{idris}|Bool| no parece ser un tipo suma por la forma en que está conformado, sin embargo,
  utilizando el tipo \mintinline{idris}|Unit|, se puede ver que es la unión etiquetada
  de \mintinline{idris}|Unit| consigo mismo.

  \begin{listing}[H]
    \begin{center}
      \begin{minipage}{0.42\textwidth}
          \begin{minted}[fontsize=\scriptsize, autogobble, stripnl=false]{idris}
          data Bool' = T () | F ()

          -- Isomorfismo Bool y Bool'
          from : Bool -> Bool'
          from True  = T ()
          from False = F ()

          to : Bool' -> Bool
          to (T _) = True
          to (F _) = False
          \end{minted}
      \end{minipage}
    \end{center}
    \caption{Isomorfismo Bool}
    \label{lst:isomorfismo}
  \end{listing}
\end{frame}

\begin{frame}[fragile]
  \frametitle{Unit}

  \mintinline{idris}|Unit| es un tipo de datos primitivo que sólo tiene un constructor
  que no puede contener información.

  El tipo de datos \mintinline{idris}|Unit| se define como sigue:

  \begin{listing}[H]
    \begin{center}
      \begin{minipage}{0.42\textwidth}
          \begin{minted}[fontsize=\scriptsize, autogobble, stripnl=false]{idris}
          data () = ()
          -- Alternativamente
          data Unit = Unit
          \end{minted}
      \end{minipage}
    \end{center}
    \caption{Tipo de datos Unit}
    \label{lst:unitdatatype}
  \end{listing}

  Estas definiciones son especiales en Idris y no se pueden replicar desde el código
  del usuario.

\end{frame}

\begin{frame}[fragile]
  \frametitle{Unit}

  Finalmente, es importante resaltar que las uniones etiquetadas de tipos se pueden
  realizar entre cualesquiera tipos.

  \begin{listing}[H]
    \begin{center}
      \begin{minipage}{0.6\textwidth}
          \begin{minted}[fontsize=\scriptsize, autogobble, stripnl=false]{idris}
          data Numero = NS String | NI Int

          data App = Pure Int | Fn (Int -> Int)
          \end{minted}
      \end{minipage}
    \end{center}
    \caption{Unión de tipos}
    \label{lst:masunionesdetipos}
  \end{listing}
\end{frame}

\end{document}
