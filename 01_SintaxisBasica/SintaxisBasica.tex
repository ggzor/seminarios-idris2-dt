\documentclass{beamer}
\usetheme[progressbar=frametitle]{metropolis}
\setbeamertemplate{frame numbering}[fraction]
\useoutertheme{metropolis}
\useinnertheme{metropolis}
\usefonttheme{metropolis}

% Barra de progreso más ancha
\makeatletter
\setlength{\metropolis@titleseparator@linewidth}{2pt}
\setlength{\metropolis@progressonsectionpage@linewidth}{2pt}
\setlength{\metropolis@progressinheadfoot@linewidth}{2pt}
\makeatother

\usepackage[utf8]{inputenc}
\usepackage[english]{babel}
\usepackage[compatibility=false]{caption}
\usepackage[listings,theorems]{tcolorbox}
\usepackage{minted}

\usemintedstyle{friendly}

\DeclareCaptionFont{captionfont}{\fontsize{8}{8}\selectfont}
\captionsetup{font=captionfont}

\title{Programación con tipos dependientes en Idris 2}
\subtitle{Sintaxis básica I}
\author{Axel Suárez Polo}
\institute{BUAP}
\date{\today}

\begin{document}

\begin{frame}
  \titlepage
\end{frame}

\begin{frame}[t]
  \frametitle{Contenidos}
  \tableofcontents
\end{frame}

\section{Introducción}

\begin{frame}
  \frametitle{¿Qué es Idris?}

  \begin{itemize}
    \item Es un lenguaje funcional \textbf{puro} con \textbf{tipos dependientes}.
    \item Tiene un sistema de tipos similar al de \textbf{Agda}.
    \item Tiene una sintaxis y manejo de efectos similar a \textbf{Haskell}
    \item Busca ser un lenguaje práctico pero con tipos dependientes.
  \end{itemize}
\end{frame}

\begin{frame}[fragile]
  \frametitle{Un ejemplo de Idris}

  \begin{listing}[H]
    \begin{center}
      \begin{minipage}{0.6\textwidth}
        \inputminted{idris}{code/helloworld.idr}
      \end{minipage}
    \end{center}
    \caption{Hello world en Idris}
    \label{lst:helloworld}
  \end{listing}
\end{frame}

\section{Sintaxis básica}
\subsection{Tipos básicos y literales}
\subsection{Tipos compuestos básicos}

\begin{frame}[fragile]
  \frametitle{Comentarios}

  En Idris existen 3 tipos de comentarios:
  \begin{itemize}
    \item Comentarios de una sóla línea:
      \vspace{5pt}
      \begin{center}
        \begin{minipage}{0.6\textwidth}
          \begin{minted}[autogobble,linenos,mathescape,fontsize=\scriptsize]{idris}
            -- este es un comentario
            -- los comentarios comienzan con "--"
          \end{minted}
        \end{minipage}
      \end{center}
      \vspace{5pt}

    \item Comentarios multilínea:
      \vspace{5pt}
      \begin{center}
        \begin{minipage}{0.6\textwidth}
          \begin{minted}[autogobble,linenos,mathescape,fontsize=\scriptsize]{idris}
            {- este es un comentario
            que se puede expandir por múltiples líneas
            -}
          \end{minted}
        \end{minipage}
      \end{center}
      \vspace{5pt}

    \item Comentarios de documentación:
      \vspace{5pt}
      \begin{center}
        \begin{minipage}{0.6\textwidth}
          \begin{minted}[autogobble,linenos,mathescape,fontsize=\scriptsize]{idris}
            ||| Este comentario documenta una declaración
            ||| Como ejemplo, main
            main : IO ()
            main = putStrLn "Hello World"
          \end{minted}
        \end{minipage}
      \end{center}
      \vspace{5pt}

  \end{itemize}

\end{frame}

\begin{frame}[fragile]
  \frametitle{Top-level declarations}

  \begin{itemize}
    \item En Idris sólo existen declaraciones \emph{top-level} y \emph{locales}.
    \item Las declaraciones \emph{top-level} tienen la siguiente sintaxis:
  \end{itemize}

  \begin{listing}[H]
    \begin{center}
      \begin{minipage}{0.6\textwidth}
        \begin{minted}[autogobble,linenos,mathescape,fontsize=\scriptsize]{idris}
          -- <identificador> : <tipo>
          -- <identificador> = <expresión>
          x : Int
          x = 10
        \end{minted}
      \end{minipage}
    \end{center}
    \caption{Sintaxis de las declaraciones \emph{top-level}.}
    \label{lst:toplevelsyntax}
  \end{listing}

  \begin{itemize}
    \item Los \textbf{identificadores} deben coincidir.
    \item La \textbf{expresión} debe ser del \textbf{tipo} especificado.
  \end{itemize}

  \footnotetext[1]{\tiny \textbf{Diferencias con Haskell}:
  La anotación de tipo es obligatoria y
  se utiliza \textbf{\scriptsize :} en lugar de \textbf{\scriptsize ::}}
\end{frame}

\begin{frame}[fragile]
  \frametitle{Identificadores}

  \begin{itemize}
    \item Los identificadores válidos son secuencias de caracteres que:
      \begin{itemize}
        \item \textbf{Comienzan con una letra} ya sea minúscula o mayúscula.
        \item Seguido de 0 o más letras, números, apóstrofes o guiones bajos.
        \item Terminan en espacio en blanco o algún carácter que no esté
          en la lista anterior.
      \end{itemize}
    \item Identificadores válidos: \emph{x, y, var\_, Test, x''3, \alpha\xi\epsilon\lambda}
    \item Identificadores inválidos: \emph{3x, x-y, \_ok, ?ok}
  \end{itemize}

  \footnotetext[1]{\tiny \textbf{Diferencias con Haskell}:
    Las variables también pueden iniciar con mayúsula sin ser tipos. }
\end{frame}

\begin{frame}[fragile]
  \frametitle{Tipos}

  Los tipos son más complejos y existe una gran diversidad de estos:

  \begin{itemize}
    \item Tipos
      \begin{itemize}
        \item \textbf{Primitivos}
          \begin{itemize}
            \item \textbf{Numéricos}
            \item \textbf{De texto}
          \end{itemize}
        \item Definidos por el usuario
          \begin{itemize}
            \item Records
            \item ADTs (Algebraic Data Types)
            \item GADTs
          \end{itemize}
        \item Compuestos
          \begin{itemize}
            \item \textbf{Tuplas}
            \item \textbf{Funciones}*
            \item Higher Kinded Types
          \end{itemize}
        \item Dependientes
          \begin{itemize}
            \item \Pi
            \item \Sigma
          \end{itemize}
      \end{itemize}
  \end{itemize}
\end{frame}

\begin{frame}[fragile]
  \frametitle{Expresiones}

  Las expresiones, al igual que los tipos, son bastante diversas:

  \begin{itemize}
    \item Expresiones
      \begin{itemize}
        \item Literales
          \begin{itemize}
            \item \textbf{Enteras}
            \item \textbf{Punto flotante}
            \item \textbf{Texto}
            \item Lista
            \item \textbf{Función (lambdas)}*
          \end{itemize}
        \item \textbf{Aplicación de funciones}*
          \begin{itemize}
            \item \textbf{Operadores y secciones}*
          \end{itemize}
        \item Identificadores
        \item Azúcar sintáctica
          \begin{itemize}
            \item \mintinline{idris}|let|
              , \mintinline{idris}|where|
              , \mintinline{idris}|case|
              , \mintinline{idris}|if|
              , \mintinline{idris}|do|
              , \mintinline{idris}|rewrite|
            \item Comprensiones de mónadas y \textbf{!-}notation
            \item Idiom brackets para aplicativos
          \end{itemize}
        \item Holes
      \end{itemize}
  \end{itemize}
\end{frame}

\begin{frame}
  \frametitle{Tipos numéricos}

  En Idris existen los tipos númericos a los que estamos
  acostumbrados de otros lenguajes y algunos nuevos:
  \begin{itemize}
    \item \mintinline{idris}|Int|: Entero de precisión fija, con al menos 32 bits.
    \item \mintinline{idris}|Integer|: Entero de precisión no acotada.
    \item \mintinline{idris}|Nat|: Entero sin signo de precisión no acotada.
    \item \mintinline{idris}|Double|: Número de punto flotante de doble precisión.
  \end{itemize}
\end{frame}

\begin{frame}
  \frametitle{Literales numéricas}

  Para obtener un valor de alguno de los tipos mencionados,
  podemos utilizar literales:

  \begin{itemize}
    \item Literales de enteros no-negativos para
      \mintinline{idris}|Int, Integer y Nat|:
      \begin{itemize}
        \item \mintinline{idris}|202103 -- decimal|
        \item \mintinline{idris}|0b0101 -- binario|
        \item \mintinline{idris}|0o2321 -- octal|
        \item \mintinline{idris}|0xefff -- hexadecimal|
      \end{itemize}
    \item Literales de enteros para
      \mintinline{idris}|Int y Integer|:
      \begin{itemize}
        \item \emph{Todas las anteriores, y además con un '-' al inicio}
        \item \mintinline{idris}|-202103|
        \item \mintinline{idris}|-0xefff|
      \end{itemize}
  \end{itemize}
\end{frame}

\begin{frame}
  \frametitle{Literales numéricas (continuación)}

  \begin{itemize}
    \item Literales de números de punto flotante para
      \mintinline{idris}|Double|:
      \begin{itemize}
        \item \emph{Todas las anteriores}
        \item \mintinline{idris}{2021.0312   -- decimal con punto}
        \item \mintinline{idris}{0.2021e-3  -- científica}
      \end{itemize}
  \end{itemize}
\end{frame}

\begin{frame}[fragile]
  \frametitle{Ejemplos literales numéricas}
  \begin{columns}[T]

    \begin{column}{.38\textwidth}
      \begin{listing}[H]
        \begin{center}
          \begin{minipage}{0.6\textwidth}
              \begin{minted}[fontsize=\scriptsize, autogobble]{idris}
              x : Int
              x = 10

              y : Double
              y = 12.2

              z : Nat
              z = 0b10101
              \end{minted}
          \end{minipage}
        \end{center}
        \caption{Literales numéricas en Idris}
        \label{lst:litnumidris}
      \end{listing}
    \end{column}


    \hspace{-0.05\textwidth}
    \vrule
    \hspace{0.025\textwidth}

    \begin{column}{.58\textwidth}
      \begin{listing}[H]
        \begin{center}
          \begin{minipage}{\textwidth}
              \begin{minted}[fontsize=\scriptsize, autogobble, breaklines]{java}
              int x = 10;


              double y = 12.2;


              // Más cercano
              var z = BigInteger.valueOf(0b10101);
              \end{minted}
          \end{minipage}
        \end{center}
        \caption{Literales numéricas en Java}
        \label{lst:litnumjava}
      \end{listing}
    \end{column}

  \end{columns}
\end{frame}

\begin{frame}
  \frametitle{Tipos de texto}

  También existen los tipos para manejar texto a los que estamos acostumbrados:

  \begin{itemize}
    \item \mintinline{idris}|Char|: Carácter Unicode (code point)
    \item \mintinline{idris}|String|: Secuencia de longitud fija de
      \mintinline{idris}{Char}'s
  \end{itemize}

\end{frame}

\begin{frame}[fragile]
  \frametitle{Literales de texto}

  Al igual que los tipos numéricos, los tipos de texto tienen literales:

  \begin{itemize}
    \item Literales de carácter para
      \mintinline{idris}|Char|:
      \begin{itemize}
        \item \mintinline{idris}|'a'    -- carácter|
        \item \mintinline{idris}|'λ'    -- carácter no ascii|
        \item \mintinline{idris}|'\97'  -- literal decimal|
        \item \mintinline{idris}|'\x61' -- literal hexadecimal|
      \end{itemize}
    \item Literales de cadena
      \mintinline{idris}|String|:
      \begin{itemize}
        \item \mintinline{idris}|"letras" -- secuencias de caracteres|
        \item \mintinline{idris}|"letr\97s"|
        \item
          \begin{minted}[autogobble]{idris}
            " multiline
            string
            " -- pueden ocupar múltiples líneas
          \end{minted}
      \end{itemize}
  \end{itemize}
\end{frame}

\begin{frame}[fragile]
  \frametitle{Tipos compuestos: Tuplas}

  \begin{itemize}
    \item El tipo compuesto más simple es la \textbf{tupla}.
    \item Una \textbf{tupla} es una colección ordenada de tamaño fijo
      que puede ser heterogénea.
    \item La sintaxis para las tuplas en Idris es una lista de 2 o más valores
      separados por comas y rodeados por paréntesis.

    \begin{listing}[H]
      \begin{center}
        \begin{minipage}{0.7\textwidth}
            \begin{minted}[fontsize=\scriptsize, autogobble]{idris}
            persona : (String, Double)
            persona = ("Juan", 9.8)

            -- Pueden tener más de 2 valores
            fecha : (Nat, Nat, Nat)
            fecha = (12, 3, 2021)

            -- Pueden anidarse y el anidamiento es significativo
            exposición : ((String, Double), (Nat, Nat, Nat))
            exposición = (persona, fecha)
            -- exposición = (("Juan", 9.8), (12, 3, 2021))
            \end{minted}
        \end{minipage}
      \end{center}
      \caption{Tuplas en Idris}
      \label{lst:tupidris}
    \end{listing}
  \end{itemize}

\end{frame}

\begin{frame}[fragile]
  \frametitle{Tipos compuestos: Tuplas (continuación)}

  \begin{itemize}
    \item Hay propiedades que no tienen las tuplas pero uno podría inferir
      \emph{incorrectamente} que tienen:
      \begin{itemize}
        \item \textbf{No conmutan en general:}
          \mintinline[fontsize=\scriptsize]{idris}{(String, Double) /= (Double, String) }
        \item \textbf{No asocian:}
          \begin{minted}[fontsize=\scriptsize,autogobble]{idris}
            (String, (Double, Nat)) /= ((String, Double), Nat)
                                    /= (String, Double, Nat)
          \end{minted}
      \end{itemize}
    \item Sin embargo, los valores de estos tipos contienen la misma cantidad de
      información, con lo que se pueden construir \textbf{isomorfismos} entre ellos.
      \vspace{4pt}
      \begin{minted}[fontsize=\scriptsize,autogobble]{idris}
      (String, Double) ~ (Double, String)
      ("Juan", 9.8) ~ (9.8, "Juan")
      \end{minted}
  \end{itemize}

\end{frame}

\begin{frame}[fragile]
  \frametitle{Tipos compuestos: Funciones}

  \begin{itemize}
    \item Uno de los tipos compuestos más útiles y centrales a la programación
      funcional es el de \textbf{tipo de función}.
    \item Un \textbf{tipo de función} está compuesto de uno o más
      \emph{tipos de entrada} y un único \emph{tipo de salida}.
    \item La sintaxis para representar un tipo de función es mediante una lista ordenada
      de dos o más tipos separados por \mintinline[fontsize=\scriptsize]{idris}{->}
      \\\scriptsize(un guión medio y un signo de mayor que).

    \begin{listing}[H]
      \begin{center}
        \begin{minipage}{0.6\textwidth}
            \begin{minted}[fontsize=\scriptsize, autogobble]{idris}
            cuadrado : Double -> Double

            edad : (String, Nat) -> Nat

            sumar : Int -> Int -> Int

            dividirEn : Int -> String -> (String, String)
            \end{minted}
        \end{minipage}
      \end{center}
      \caption{Tipos de función en Idris}
      \label{lst:funidris}
    \end{listing}
  \end{itemize}

\end{frame}

\begin{frame}[fragile]
  \frametitle{Tipos compuestos: Funciones (continuación)}

  \begin{itemize}
    \item Al igual que con las tuplas, los tipos de función tienen algunas propiedades:
      \begin{itemize}
        \item \textbf{No conmutan en general:}\\
          \mintinline[fontsize=\scriptsize]{idris}{String -> Int /= Int -> String}
        \item \textbf{Asocian sólo a la derecha:} A diferencia de las tuplas, los tipos
          de función si presentan asociatividad pero sólo hacia la derecha:\\
          \begin{onlyenv}<2>
            \begin{minted}[fontsize=\scriptsize,autogobble]{idris}
          --I     I      O      I       I      O
            \end{minted}
          \end{onlyenv}
          \begin{minted}[fontsize=\scriptsize,autogobble]{idris}
          Int -> Int -> Int == Int -> (Int -> Int)
          \end{minted}
          \begin{onlyenv}<2>
            \begin{minted}[fontsize=\scriptsize,autogobble]{idris}
          --I     I      O      I       O      O
            \end{minted}
          \end{onlyenv}
          \begin{minted}[fontsize=\scriptsize,autogobble]{idris}
          Int -> Int -> Int /= (Int -> Int) -> Int
          \end{minted}
          Una forma de no perderse es recordar que los tipos de función ``marcan"
          a los tipos como de entrada o de salida, por lo que un tipo de entrada nunca
          puede convertirse en uno de salida.
      \end{itemize}
  \end{itemize}

\end{frame}

\begin{frame}[fragile]
  \frametitle{Tipos compuestos: Funciones (continuación)}

  \begin{itemize}
    \item También presentan \textbf{isomorfismos} que más adelante serán vistos:

    \vspace{4pt}
    \begin{minted}[fontsize=\scriptsize,autogobble]{idris}
    -- atestiguado por flip
    Int -> Double -> String  ~  Double -> Int -> String
    -- atestiguado por curry y uncurry
    (Int, Double) -> String  ~  Int -> Double -> String
    \end{minted}
  \end{itemize}

\end{frame}

\begin{frame}[fragile]
  \frametitle{Funciones sobre tipos numéricos}

  \begin{itemize}
    \item Idris incluye por defecto múltiples funciones para trabajar sobre tipos
      numéricos:
      \vspace{4pt}
      \begin{minted}[fontsize=\scriptsize,autogobble]{idris}
      (+) : Int -> Int -> Int
      (*) : Int -> Int -> Int
      \end{minted}
    \item Estas mismas funciones existen para los demás tipos numéricos:
      \mintinline[fontsize=\scriptsize]{idris}{Nat, Integer y Double}.
    \item También se incluyen la función de resta, pero no está definida para
      \mintinline[fontsize=\scriptsize]{idris}{Nat}.
      \vspace{4pt}
      \begin{minted}[fontsize=\scriptsize,autogobble]{idris}
      (-) : Int -> Int -> Int
      \end{minted}
  \end{itemize}

\end{frame}

\begin{frame}[fragile]
  \frametitle{Funciones sobre tipos numéricos (continuación)}

  \begin{itemize}
    \item Además, define la operación de división entera y módulo para los tipos
      enteros (\mintinline[fontsize=\scriptsize]{idris}{Int e Integer}):
      \vspace{4pt}
      \begin{minted}[fontsize=\scriptsize,autogobble]{idris}
      div : Int -> Int -> Int
      mod : Int -> Int -> Int
      \end{minted}
    \item Finalmente, también define la división de números de punto flotante:
      \vspace{4pt}
      \begin{minted}[fontsize=\scriptsize,autogobble]{idris}
      (/) : Double -> Double -> Double
      \end{minted}
  \end{itemize}
\end{frame}

\begin{frame}[fragile]
  \frametitle{Aplicación de funciones}

  \begin{itemize}
    \item Lo más relevante que se puede hacer con una función es aplicarla.
    \item Como esto es lo más común de hacer, en Idris se proporciona la sintaxis de
      aplicación de función que consiste en lo siguiente:
      \begin{itemize}
        \item Empieza con una \textbf{expresión} del tipo función.
        \item Es seguida por 1 o más expresiones separadas por espacios
          en blanco, que serán los argumentos para la función.
        \item Una \textbf{expresión de aplicación función} es válida cuando:
          \begin{itemize}
            \item se proporcionan expresiones con los tipos correspondientes a los
              parámetros que declara la función.
            \item se proporciona una cantidad de argumentos \textbf{menor o igual}
              a la cantidad de parámetros que declara la función.
          \end{itemize}
      \end{itemize}
  \end{itemize}
\end{frame}

\begin{frame}[fragile]
  \frametitle{Ejemplos de aplicación funciones}
  \begin{columns}[T]

    \begin{column}{.48\textwidth}
      \begin{listing}[H]
        \begin{center}
          \begin{minipage}{0.6\textwidth}
              \begin{minted}[fontsize=\scriptsize, autogobble]{idris}
              sumar : Int -> Int -> Int
              sumar = (+)

              x : Int
              x = 10

              y : Int
              y = 20

              z : Int
              z = sumar x y -- 30

              z' : Int
              z' = sumar 10 20
              \end{minted}
          \end{minipage}
        \end{center}
        \caption{Aplicación de funciones en Idris}
        \label{lst:apfunidris1}
      \end{listing}
    \end{column}


    \hspace{-0.05\textwidth}
    \vrule
    \hspace{0.025\textwidth}

    \begin{column}{.5\textwidth}
      \begin{listing}[H]
        \begin{center}
          \begin{minipage}{\textwidth}
              \begin{minted}[fontsize=\scriptsize, autogobble, breaklines]{java}
              int sumar(int a, int b) {
                return a + b;
              }

              int x = 10;

              int y = 20;


              int z = sumar(x, y);


              int zp = sumar(10, 20);
              \end{minted}
          \end{minipage}
        \end{center}
        \caption{Aplicación de funciones en Java}
        \label{lst:apfunjava1}
      \end{listing}
    \end{column}

  \end{columns}
\end{frame}

\begin{frame}[fragile]
  \frametitle{Operadores}

  \begin{itemize}
    \item En Idris, se pueden definir operadores, que no son más que funciones pero que
      se pueden aplicar de manera \textbf{infija} o \textbf{prefija}.
    \item Ejemplos de estos operadores son las funciones aritméticas que hemos visto con
      anterioridad:
      \begin{listing}[H]
        \begin{center}
          \begin{minipage}{0.5\textwidth}
              \begin{minted}[fontsize=\scriptsize, autogobble, breaklines]{idris}
              x : Int
              x = 10
              y : Int
              y = 20

              z : Int
              z = (+) 10 20

              z' : Int
              z = 10 + 20
              \end{minted}
          \end{minipage}
        \end{center}
        \caption{Aplicación de operadores}
        \label{lst:apopidris1}
      \end{listing}
  \end{itemize}
\end{frame}

\end{document}
